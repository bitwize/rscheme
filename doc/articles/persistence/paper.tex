\documentclass[10pt,letterpaper]{article}

\usepackage[T1]{fontenc}
\usepackage{textcomp}
\usepackage{mathptmx}
\usepackage{helvet}
\usepackage{courier}

\usepackage{epsfig}

\begin{document}

\title{Persistent Object Storage for Scheme}
\author{Donovan Kolbly}
\date{\today}
\maketitle

\begin{abstract}

Persistent storage of data is crucial to many applications, yet often
the approach is to explicitly serialize and deserialize the relevant
objects of the application.  In the RScheme system, which is an
implementation of the Scheme language, we adopted a persistent object
storage approach using the pointer swizzling at page-fault time
described in [Texas].  This approach has the advantage that program
objects are loaded into memory automatically by the system as required,
using the fast page-table hardware to remove all cost of runtime access
while preserving support for destructive updates (mutation).

In this paper, we discuss the implementation and how the approach
interacts with a garbage collected and dynamically typed language.

\end{abstract}

\section{Introduction}

\section{How It Works}

A. When a pstore is created, this is what happens:

1. create the file on disk

2. create a root allocation area, which in turn allocates a single
   page in memory.  The act of initializing the AA will mark the
   page dirty.

[goto C]

B. When a pstore is opened, this is what happens:

1. open the file on disk.  By default, write access is obtained and
   the file is locked for exclusive access using flock() or a similar
   OS primitive.

2. read record 0 and extract the

[goto C]

C. Here is how things proceed:

1. maintain a page info table.  Each page is either RESERVED, LOADED,
   or DIRTY.  In these cases, the page is respectively READ-PROTECTED,
   WRITE-PROTECTED, or UNPROTECTED.  Keep these in three sets with
   mutual containment (i.e., reserved >= loaded >= dirty).

2. when a read fault happens (presumably to a RESERVED page), load the
   page from disk and mark it LOADED

3. when a write fault happens (presumably to a LOADED page), mark the
   page DIRTY.

4. when a commit is invoked, create a Moved Objects table.  Then, as long
   as there are DIRTY pages, try to write one out [call D].  When there
   are no dirty pages left, write out the commit record.

D. Writing out a page (in the context of a Moved Objects table)

1. A page is either a page of small objects, the first page of a large
   object, or an interior page of a large object.  Use that information
   to identify the framing of the page and which memory locations contain
   scheme objects (potential pointers)

   Throughout a single page, we maintain a Needs Moving list.

2. Process the page in two passes.  The first pass is a scan, in which
   a table of all the referenced pages is assembled.  The second pass
   is actual serialization of the page, starting with a representation
   of the referenced-pages list, followed by the objects themselves.
   (During the first pass, we actually remember the referenced pages
   in slot-traversal order so that during the second pass we don't have
   to recompute this information.)

   Between the two passes, the act of assembling the table of referenced
   pages assigns ``local'' page numbers, which are essentiall indexes into
   the table.


3. When a pointer is encountered, one of the following pertains:

3a. The pointer is a key in Moved Objects.  In this case, replace
    the pointer with the value and try again.

3b. The pointer is a key in the Pivot table.  In this case, write out
    the value in the Pivot Id, which is as follows:

\begin{verbatim}
             24                           6         2
+----------------------------------+-------------+-----+
|    Virtual Page Number           |   Index     | 1 1 |
+----------------------------------+-------------+-----+
\end{verbatim}

3c. The pointer is onto a persistent page.  In this case, we simply
    reference the object.


3d. The pointer is onto a non-persistent (transient) page.  In this
    case, we add the object to the Needs Moving list.

4. When finished with the first pass, if the Needs Moving list is
   empty we proceed to the second pass in which we serialize the
   data and write it to the underlying LSS.

5. If the Needs Moving list is not empty, then we have identified
   some objects which are to be copied into the store.  We return
   this list to the scheme level, where it is interpreted as an
   unsuccessful commit attempt.  The Scheme-side implementation
   traverses the list, copying the objects to the store and inserting
   the appropriate entries in the Moved Objects table.

   The Scheme-side also extends the symbol table list when it sees a
   symbol in the Needs Moving list.  In this case, of course, instead
   of copying the symbol into the store it adds an entry in an
   indirect page for symbols.  On the next pass, then, scanning the
   symbol will go down the Pivot table branch.


Serialized Representation of a Pointer:

\begin{verbatim}
             19                           11        2
+----------------------------------+-------------+-----+
|    Local Page Number             |   Offset    | 1 1 |
+----------------------------------+-------------+-----+
\end{verbatim}

Although the pointer format contains only 19 bits, this is more than
enough because there cannot possibly be more than 2048 different pointers
referenced from a single 8K page with 32-bit words (i.e., if every
pointer pointed into a different page, you would get that many).
The Local Page Index, on the other hand, has a full 32 bits to refer
to the target page.  Hence, up to 4G pages of 8K can be referenced
in the current implementation, corresponding to 32TB of persistent
virtual address space.



Note that since the VPN is only 24 bits, pivot references are limited
to pages 0 - 0xFF,FFFF.


\section{First-Class Memory Allocation Areas}

In non-garbage-collected languages, heap allocation is fairly
explicit.  This happens because, in the absence of garbage collection,
the application must pay close attention to objects being allocated
because it will have to free them up later.  The Texas system exploited
this fact and required each allocation site to name the particular
allocation area from which the new object should be obtained.

In contrast, Scheme and other GC'd languages more often have standard
procedures that allocate memory on behalf of the application.  For
example, the Scheme map procedure allocates pairs which are returned
to the application.  It would be impractical to have ``persistent''
versions of all such procedures, and would require embedding knowledge
of persistence deeply into the application (and system) code.


Because of this widespread and implicit allocation, specifying the
appropriate allocation area explicitly at each call site is
impractical.

RScheme addresses this issue by implementing a dynamically scoped
``current allocation area''.  This feature is part of the base system
(and may be used to implement the memory allocation aspect of
sandboxes), but using the persistent store adds a new kind of allocation
area which is a persistent allocation area.

The term ``area'' does not imply that the pages are contiguous.
In virtual memory, pages obtained from the operating system to store
persistent and transient objects may be intermixed, and pages
from different allocation areas within a single store may
also be intermixed.

\subsection{Hierarchical Allocation Areas}


\section{Reachability-Based Persistence and Losing Object Identity}

In some cases, it is not convenient even to change the default allocation
area either because it is not even known at the time allocation takes
place whether or not the object is destined for the persistent heap.
In general, this can be handled by GCing the persistent heap.  Nevertheless,
a high persistent allocation rate is far more difficult to deal with
than an equal dynamic heap allocation rate.

For this reason, the pstore employs reachability-based persistence as
a fallback mechanism to employ when a pointer to a transient object is
encountered in attempting to write out a page.

RScheme employs a non-moving [cite GC paper] garbage collector.  Since
objects must be placed on persistent pages in order to be managed by
the system, we lose temporarily object identity for transient objects
that are reachable from persistent objects.

\section{Persistent Garbage Collection}

This section describes the garbage collection issues and implementation
involved with both the persistent object store and how it interacts
with the transient heap.  

Because the separation of the two heaps
is largely transparent to the application, the system must ensure
GC correctness in the presence of many cross-heap pointers and in
the presence of adversarial mutator (application) activity.

Correctness is defined as never reclaiming an object that could
possibly still be used by the application.  An object could possibly
be used if there is some path to it from a root.  In RScheme, the
roots consist of the virtual machine registers, the stack cache,
and the root vectors for each loadable module.

In RScheme, the transient and persistent garbage collections are both
\emph{incremental}.  This means that the application
continues to run while GC activity happens behind the scenes.  This
greatly complicates the implementation (especially in the presence of
two heaps!)  but provides the considerable benefit of allowing the
application to avoid any long pauses.  In order to achieve
incrementality, RScheme uses a \emph{write barrier} to
notice when application activity might interfere with the correctness
of an ongoing garbage collection cycle.

A persistent store does not, by default, have garbage collection
enabled for it.  To turn on GC for a persistent store, call
<code>(start-online-compacter <replaceable>pstore</replaceable>)</code>.
Online compaction and GC is only applicable to pstores that have two
LSS volumes, because part of the process involves compacting the
underlying LSS.

\subsection{Interaction with Transient GC}

It is possible for an uncommited persistent object store to hold the
only pointer to a transient object.  Hence, correctness requires that
the system keep track of pointers from persistent objects into the
transient heap.  This type of tracking is typical for any generational
garbage collector, and these kinds of pointers represent
\emph{intergenerational pointers}.  Here, the
uncommitted state of the persistent store acts as an older generation
which may contain pointers into a younger generation.

As an example, consider Figure \ref{fig.intergen}.  Persistent
object f contains the only reference to transient object h, and
persistent object k the only reference to transient object l.  If the
system did not notice that f has a pointer to h, then the storage for
h might be reclaimed as garbage.  A subsequent traversal through e and
f would likely crash the system.

\begin{figure}
\centering
\epsfig{file=intergen.pdf}
\caption{Data Structures Spanning Transient and Persistent Heaps}
\label{fig.intergen}
\end{figure}

\begin{figure}
\centering
\epsfig{file=extraheapptrs.pdf}
\caption{Use of the \texttt{extraHeapPointers} Set}
\end{figure}


\section{Symbols, Classes, and ``Program'' Objects}

Objects that are to be considered part of the program and not
part of the data must be marked in a special way so to exempt
them from the reachability-based persistence traversal.

Furthermore, they must be given names that are consistent among
different attempts to open the store.


RScheme handles this by requiring that the application assign each
such special object an integer value, which is used to refer the
object in the store.

In the current implementation, the namespace is
actually organized as virtual persistent pages.

Furthermore, by default, the system supplies a set of such
mappings for common classes such as <pair> and <string>.  These
defaults allow an application to persist standard scheme data
objects without any special setup.

\begin{verbatim}
(define *p* (create-pstore "foo.sto"))
(commit *p* (vector 1 'a (2 3 4) "foo"))
\end{verbatim}

\begin{figure}
\centering
\epsfig{file=symindirpp.pdf}
\caption{Format of Symbol Indirect Page Payload}
\end{figure}

In addition there are two words of header data introduced by the
indirect page framework:

\begin{figure}
\begin{verbatim}
+---------------+---------------+-----/ /---
|    type id    |  instance id  |   payload
+---------------+---------------+-----/ /---
\end{verbatim}
\caption{Indirect Page Overhead Data}
\end{figure}

The type id is used so that different sorts of indirect pages may
exist.  The instance id is an extra word that the type-specific
processor can utilize.

This mechanism, for example, would allow us to store linker information
so that general code pointers could be stored.

Currently, only type id 0 is assigned and refers to a symbol vector,
and it always uses instance id 0.

Unfortunately, because translation of addresses from persistent
to transient addresses happens at page fault time, it is impossible
in RScheme to have a full Scheme-level callback to handle these
translations.  New indirection types must have their unswizzling
handlers implemented in C code.  Other scheme implementations and
other langauges may not have this limitation.

Forward indirect pointer binding (swizzling of new pivotal objects
like symbols) can happen at the scheme level, and does in the current
implementation.

[optional...] This problem arises because RScheme uses /precise/
garbage collection (i.e., not stack scanning) only at ``safe points''.
Since a page fault could happen at any time, it might not be a safe
point so invoking scheme code to handle it would not be allowed.
[This is a general problem with RScheme's interaction with C-level
code.  It may be possible to do a trick that turns off all GC except
at top-level safe points, but this may have its own issues.  In
particular, lots of low-level code in RScheme assumes that monotones
are atomic and uses this fact to avoid locking.]


\section{Persistent Page Numbers}

VPPs appear in the address space of normal pages.  Each such VPP can
hold up to 64 objects.  The system reserves 10 pages for itself (page
numbers 0 through 9) to supply a default set of pivot objects.  Pages
10 through 255 are available for use by the user application.  Pages
256 through 0xFF,FFFF are used for dynamically created pivots such as
symbols.  The system starts allocating persistent page numbers at
0x100,0000




\section{Log-structured Storage}

The RScheme system builds the persistent object store atop a
log-structured store (which was actually developed specifically to
support the pstore, although it can be used independently.)
The LSS exports a model of records, identified by unsigned 32-bit
integers, read and write operations, and a commit operation.
The pstore maps virtual page numbers directly to record numbers.
Since VPPs up to 0x100 do not have any correpsonding data to
store, record numbers 0-255 are available for other purposes.  In the
current implementation, we use record number 0 to store the
pstore-level commit information.

pstore-level commit info:

  \texttt{next\_page\_num}
  \texttt{next\_indirect\_page}
  \texttt{root\_area}



Useful for building persistent hash tables.

Returns the normal transient hash code for objects which are
transient, and in this sense is an elaboration or extension of the
transient->hash primop.

Note that RScheme does not have a default implementation of
hash-code (i.e., a method on <object>), and even if it did
(in the spirit of the default behavior of equal? being eq?),
loading the pstore module would not suggest overriding that
behavior to use hash-code/persistent because the latter
is much more expensive.

\section{Issues}

\subsection{Escaping Pointers}

In theory, functions and closures can be written out, but in practice
it is quite difficult to get right.  The use of reachability-based
persistence together with a split heap (persistent vs. transient)
implies it is very difficult to avoid pointer escape when writing
functions.  Also, code contains pointers into the program memory.
In the current implementation, code pointers must be annotated explicitly
in a Pivot table or else they will not be translated.  Only the code
pointers for the bytecode interpreter are stored by default.

Consider, for example, the procedure (lambda (a) (+ a 1)).  Behind the
scenes is a pointer to the `+' procedure (actually, in the default
environment of RScheme, this will contain a pointer to the `+'
top-level variable object because side-effecting + is allowed.)  In
addition, the compiler may insert various other references for debugging
or optimization support.  For example, in RScheme simply writing a lambda
at the top level creates a closure which contains a pointer to the current
top-level environment which is used for optimizing the lambda.)


In RScheme, we have therefore found it convenient to have the
Needs Moving processor check to see if the object to be moved
is probably an inadvertant escaping pointer, such as a <<class>>
object.  If such an object is encountered, an error will
be signalled.

   At the Scheme level, some kinds of objects can be marked as ``NO COPY''
   as a debugging hint that they should never be copied.  For example,
   by default <<class>> instances (i.e., classes) are NO COPY because
   generally it is a bad idea when a class gets written out.

(1) cleanup so a failure commit doesn't leave things broken, e.g.,
    in case of class escape

(2) commit of

        top[1]=>(define (f x) `(1 ,x 1))
        value := f
        top[2]=>(commit p f)


hangs!


Improvements

(1) fast-copy class list

(2) build symbol set fast

(3) use symbol set across more than one commit currently, because the
    scope of the New Symbols list is the copy-in operation, we blow up
    to one page per (dirty page per commit, up to 1/(64*N) as
    efficient!)

(4) paging out from pstore instead of letting OS do it (LOADED -> RESERVED)

(5) reclaiming VM (RESERVED -> FREE)

(6) pivots with structured names instead of integers.

(7) concurrency... multiple read OK, multiples writers or mixed BAD.  commit
    may count as writing because it mutates things as it moves them.  Although,
    note that if nothing needs to be copied in, commit is atomic with respect
    to RScheme threads.


I can and plan on doing (1)-(3) by the next major RS release
(0.7.3.5).  (4) is not too hard, but (5) is quite difficult because of
the GC interactions [i.e., we have to make sure that we don't unmap a
page that is referenced by the transient heap.  This might turn out to
be easy if we just notice that a page [that is not the body of a large
object] that is RESERVED for a complete GC cycle can be unmapped.
This works because the act of traversal will touch a page when the GC
reads the header bits during scanning of the referencing object.]

(6) is doable but harder than (4).



\section{Appendix A.  Name Rationalization...}

The symbol names used in the current implementation are actually
somewhat quirky.  For clarity, this paper uses rationalized names.
This table maps the names used in this paper to the actual
implementation.

\begin{tabular}{|l|l|}
\hline
current (RS 0.7.3.4-b3) name    &       name in paper (if different) \\
\hline
\texttt{<persistent-store> }    &  \\
\texttt{open-persistent-store } &       \texttt{open-pstore} \\
\texttt{create-persistent-store }       & \texttt{create-pstore} \\
\texttt{open-pstore-on-lss }    & \\
\texttt{close-pstore }  &       \texttt{close-pstore} \\
\hline

\texttt{root-object }   &       \texttt{pstore-root} \\
\texttt{commit }        &       \texttt{pstore-commit} \\
\texttt{num-dirty-pages }       &       \texttt{pstore-dirty-page-count} \\
\texttt{default-allocation-area }       &       \texttt{pstore-root-allocation-area} \\
\texttt{object->allocation-area }       &       \texttt{object-allocation-area} \\
\texttt{make-allocation-area }  & \\
\texttt{register-indirect-page }        &       \texttt{pstore-set-virtual-page-contents} \\
\hline

\texttt{pstore-write-protect? } &       \texttt{pstore-write-protected?} \\
\texttt{pstore-set-write-protect! }     &       \texttt{pstore-set-write-protected!} \\
\hline

\texttt{transient->persistent } &       \texttt{object->paddr} \\
\texttt{persistent->transient } &       \texttt{paddr->object} \\
\texttt{pstore-last-commit-id } &       \texttt{pstore-snapshot} \\
\texttt{pstore-set-commit-id }  &       \texttt{pstore-set-snapshot!} \\
\hline


\texttt{persistent-hash }       &       \texttt{hash-code/persistent} \\
\hline

\end{tabular}

\begin{figure}
\centering
\epsfig{file=gcphases.pdf}
\caption{Phases of Persistent Garbage Collection}
\end{figure}

\begin{figure}
\centering
\epsfig{file=pagesinmem.pdf}
\caption{Pages In Memory}
\end{figure}


\end{document}
